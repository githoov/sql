\documentclass[mathserif, xcolor=table, pdfpages, default, graphics, listings]{beamer}
\usepackage[utf8]{inputenc}
    \mode<presentation>
\usepackage{beamerthemesplit, amssymb, verbatim, listings}
    \usefonttheme{serif}
\usepackage[labelformat=empty]{caption}

\title{Manipulating Date Fields and An Introduction to Joins} 
\author{Scott Hoover} 
\date{\today}

\begin{document}

    \frame{\titlepage} 

    \section{Outline}

        \begin{frame}
        \begin{itemize}
            \item Manipulating Date Fields
            \pause \item The Concept of Joins
            \pause\item SQL and its {\tt{JOIN}} Operations
        \end{itemize}
        \end{frame}

    \section{Manipulating Date Fields}

        \frame{When working with data, time is typically an important variable. It's common to restrict one's analysis to a certain date range, or summarize data by some date field (\emph{e.g.}, month, year, day of week, or quarter). Knowing how to manipulate dates is important.}

        \frame{The SQL functions we will use to manipulate date fields are {\tt{BETWEEN}}, {\tt{EXTRACT}}, {\tt{DATE\_FORMAT}}, {\tt{DATE\_ADD}}, and {\tt{INTERVAL}}}


    \section{The Concept of Joins}


        \subsection{A Set Theory Primer}
        \begin{frame}
        {Intersection}
            \begin{align*}
                A &= \{1,3,5,7,9\} \\
                B &= \{2,4,6,8,10\} \\
                A \cap B &= \varnothing
            \end{align*}
        \end{frame}

        
        \begin{frame}
            \begin{align*}
                A &= \{1,2,3,4\} \\
                B &= \{3,4,5,6\} \\
                A \cap B &= \{3,4\}
            \end{align*}
        \end{frame}   


        \frame{The notion of intersection translates to joins, in that the \emph{x} tables being joined only return rows and columns in which there is a shared id or joining variable. In SQL, this is the {\tt{INNER JOIN}} operation. Let's see this with Facebook users and status updates.} 


        \begin{frame}
        {Union}
            \begin{align*}
                A &= \{1,3,5,7,9\} \\
                B &= \{2,4,6,8,10\} \\
                A \cup B &= \{1,2,3,4,5,6,7,8,9,10\}
            \end{align*}
        \end{frame}     


        \begin{frame}
            \begin{align*}
                A &= \{1,2,3,4\} \\
                B &= \{3,4,5,6\} \\
                A \cup B &= \{1,2,3,4,5,6\}
            \end{align*}
        \end{frame} 


        \begin{frame}
        {Compliment}
            \begin{align*}
                U &= \{1,2,3,4,5,6,7,8,9,10\} \\
                A &= \{1,3,5,7,9\} \\
                A'&= \{2,4,6,8,10\}
            \end{align*}
        \end{frame}             

        \frame{The notion of compliment and intersection translate into joins in SQL with {\tt{FULL OUTER JOIN}}, {\tt{LEFT OUTER JOIN}}, and {\tt{RIGHT OUTER JOIN}}}

\end{document}